\section{Appendix A: Separation Distance from Second Moments}\label{sec:appA}

From the LSST catalogs the following {\tt Object} table elements are used to define the parameters needed to calculate the separation distance \citedsp{LSE-163}:

\begin{center}
\begin{tabular}{ccll}
\hline
Parameter & Unit & Table Element & Description \\
\hline
$x_{\rm trans},y_{\rm trans}$ & $\rm degrees$ & {\tt DIAObject} {\tt radec} & transient centroid  \\
$x_{\rm gal},y_{\rm gal}$       & $\rm degrees$ & {\tt Object} {\tt radec}       & galaxy centroid      \\
$\overline{x^2}$,$\overline{y^2}$,$\overline{xy}$ & $\rm arcsec^2$  & {\tt Object} {\tt Ixx}, {\tt Iyy}, {\tt Ixy} & galaxy second moments \\
\hline
\end{tabular}
\end{center}

There might be an issue with using the {\tt Object} catalog second moments: the {\tt Ixx}, {\tt Iyy}, and {\tt Ixy} are defined with with respect to the local tract/patch and not sky coordinates (Jira DM-19519).
Although the local tangent projection will probably work fine for this application of the second moments, this should be verified at the time of implementation.

As described in Section 10 of E. Bertin's Source Extractor manual\footnote{Version 2.3: \url{https://www.astromatic.net/pubsvn/software/sextractor/trunk/doc/sextractor.pdf}} (and presumably many other places), the unitless ellipse parameters $C_{xx},C_{yy},C_{xy}$ can be calculated from the second moments via:

\begin{equation}
C_{xx} = \frac{\overline{y^2}}{\sqrt{ \left( \frac{\overline{x^2}-\overline{y^2}}{2} \right)^2 + \overline{xy}^2}}
\end{equation}

\begin{equation}
C_{yy} = \frac{\overline{x^2}}{\sqrt{ \left( \frac{\overline{x^2}-\overline{y^2}}{2} \right)^2 + \overline{xy}^2}}
\end{equation}

\begin{equation}
C_{xy} = -2 \frac{\overline{xy}}{\sqrt{ \left( \frac{\overline{x^2}-\overline{y^2}}{2} \right)^2 + \overline{xy}^2}}
\end{equation}

The sky distances between the transient and galaxy centroids are calculated as follows, and include the $\cos(\delta)$ factor and a conversion from units of degrees to arcseconds:

\begin{equation}
x_r = 3600(x_{\rm SN} - x_{\rm gal})
\end{equation}

\begin{equation}
y_r = 3600(y_{\rm SN} - y_{\rm gal})\cos{y_{\rm gal}}
\end{equation}

Finally, the separation distance $R$ in arcseconds is given by:

\begin{equation}
R^2 = C_{xx} x_r^2 + C_{yy} y_r^2 + C_{xy} x_r y_r.
\end{equation}
